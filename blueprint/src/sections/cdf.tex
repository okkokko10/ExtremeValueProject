\chapter{Cumulative distribution functions}

\begin{definition}
  \label{def:cdf}
  \lean{CumulativeDistributionFunction}
  \leanok
  A function $F \colon \R \to \R$ is a
  \emph{cumulative distribution function} (c.d.f.) if
  \begin{enumerate}
  \item[(i)] $x \mapsto F(x)$ is increasing;
  \item[(ii)] $x \mapsto F(x)$ is right-continuous;
  \item[(iii)] $\lim_{x \to -\infty} F(x) = 0$ and $\lim_{x \to +\infty} F(x) = 1$.
  \end{enumerate}
\end{definition}

\begin{lemma}
  \label{lem:cdf-of-random-var}
  \uses{def:cdf}
  \lean{MeasureTheory.ProbabilityMeasure.cdf}
  \leanok
  If $X$ is a real-valued random variable, then the function
  $F \colon \R \to \R$ given by $F(x) = \PR \big[ X \le x \big]$ is a c.d.f.
\end{lemma}
\begin{proof}
  % \leanok
  Property (1.) in Definition~\ref{def:cdf} is obvious (by monotonicity of measures)
  and properties (2.) and (3.) are simple consequences of monotone convergence
  theorems for probability measures.
\end{proof}

\section{Degenerate distributions}

\begin{definition}
  \label{def:degenerate-cdf}
  \uses{def:cdf}
  \lean{CumulativeDistributionFunction.IsDegenerate}
  \leanok
  A c.d.f. $F$ is said to be
  \emph{degenerate} if for every $x \in \R$ we have either $F(x) = 0$ or $F(x) = 1$.
  Otherwise $F$ is said to be \emph{nondegenerate}.
\end{definition}

\begin{lemma}
  \label{lem:degenerate-cdf-iff-exists-jump}
  \uses{def:degenerate-cdf}
  \lean{CumulativeDistributionFunction.isDegenerate_iff}
  \leanok
  $F$ is a degenerate c.d.f. if and only if there exists a $x_0 \in \R$ such that
  \begin{align*}
  F(x) = \begin{cases}
          0 & \text{ for } x < x_0 \\
          1 & \text{ for } x \ge x_0 .
         \end{cases}
  \end{align*}
\end{lemma}
\begin{proof}
  \uses{}
  \leanok
  The ``if'' direction is clear. To prove the ``only if'' direction, assume that
  $F$ is a degenerated c.d.f., and let $x_0 = \inf \set{ x \in \R \; \big| \; F(x) = 1}$.
  Then it is straightforward to show by properties of a c.d.f. that $F$ has the asserted form.
\end{proof}

\begin{lemma}
  \label{lem:delta-has-degenerate-cdf}
  \uses{def:degenerate-cdf}
  \lean{CumulativeDistributionFunction.diracProba_is_degenerate}
  \leanok
  The c.d.f. of Dirac delta mass $\delta_{x_0}$ at $x_0 \in \R$ is degenerate.
\end{lemma}
\begin{proof}
  \uses{lem:degenerate-cdf-iff-exists-jump}
  \leanok
  \ldots
\end{proof}

\begin{lemma}
  \label{lem:degenerate-cdf-is-delta}
  \uses{def:degenerate-cdf}
  \lean{CumulativeDistributionFunction.eq_diracProba_of_isDegenerate}
  \leanok
  If a c.d.f. $F$ is degenerate, then it is the c.d.f. of a Dirac delta mass $\delta_{x_0}$
  at some point $x_0 \in \R$.
\end{lemma}
\begin{proof}
  \uses{lem:degenerate-cdf-iff-exists-jump}
  \leanok
  \ldots
\end{proof}

\section{Distributions of maxima of independent random variables}

% \begin{lemma}
% \label{lem:random-var-of-cdf}
% \uses{def:cdf}
% If $F \colon \R \to \R$ is a c.d.f., then there exists a probability
% space $(\Omega, \mathcal{F}, \PR)$ and a real-valued random variable
% $X \colon \Omega \to \R$ such that for every $x \in \R$
% we have $F(x) = \PR \big[ X \le x \big]$.
% \end{lemma}
% \begin{proof}
% Abstract nonsense (take $\Omega = [0,1]$ and $X = \idOf{\R}$ ).
% \end{proof}

\begin{lemma}
  \label{lem:cdf-of-max-two}
  \uses{def:cdf}
  % \leanok
  Let $X$ and $Y$ be two independent real-valued random variables with respective
  cumulative distribution functions $F$ and $G$,
  i.e. $F(x) = \PR \big[ X \le x \big]$ and $G(x) = \PR \big[ Y \le x \big]$.
  Then the c.d.f. of $M = \max (X, Y)$ is $x \mapsto F(x) \, G(x)$.
\end{lemma}
\begin{proof}
  \uses{}
  % \leanok
  Fix $x \in \R$. Note that $\max (X, Y) \le x$ if and only if both $X \le x$ and $Y \le x$.
  Calculate, using independence,
  \begin{align*}
  \PR\big[ \max (X, Y) \le x \big]
  \; = \; \PR\big[ X \le x \, \; Y \le x \big]
  \; = \; \PR\big[ X \le x \big] \; \PR\big[ Y \le x \big]
  \; = \; F(x) \, G(x) .
  \end{align*}
\end{proof}

\begin{lemma}
  \label{lem:cdf-of-max-many}
  \uses{def:cdf}
  % \leanok
  Let $X_0, X_1, \ldots, X_{n-1}$ be independent identically distributed real-valued random variables
  with cumulative distribution functions $F$,
  i.e. $F(x) = \PR \big[ X_j \le x \big]$ for every $j$.
  Then the c.d.f. of \[ M_n = \max_{0 \le j < n} X_j\] is the function $x \mapsto \big(F(x)\big)^n$.
\end{lemma}
\begin{proof}
  \uses{lem:cdf-of-max-two}
  % \leanok
  Induction on~$n$ using \ref{lem:cdf-of-max-two}.
\end{proof}

\begin{lemma}
  \label{lem:cdf-of-affine-max-many}
  \uses{def:cdf}
  % \leanok
  Let $X_0, X_1, \ldots, X_{n-1}$ be independent identically distributed real-valued random variables
  with cumulative distribution functions $F$,
  i.e. $F(x) = \PR \big[ X_j \le x \big]$ for every $j$,
  and let $a > 0$ and $b \in \R$.
  Then the c.d.f. of
  \begin{align*}
  \hat{M}_n = \frac{\max_{0 \le j < n} X_j - b}{a}
  \end{align*}
  is the function $x \mapsto \big(F(a x + b)\big)^n$.
\end{lemma}
\begin{proof}
  \uses{lem:cdf-of-max-many}
  % \leanok
  Use \ref{lem:cdf-of-max-many} and do a change of variables.
\end{proof}

\section{Distributions of minima of independent random variables}

\section{Equivalence classes modulo affine transformations}

\begin{definition}
  \label{def:oriented-affine-isomorphism}
  \lean{AffineIncrEquiv}
  \leanok
  The collection of all transformations $\R \to \R$ of the form
  $x \mapsto a x + b$, where $a>0$, $b \in \R$, forms a group.
  We call this the \emph{orientation preserving affine isomorphism group}
  and denote it by $\OriAffR$.
\end{definition}

\begin{definition}
  \label{def:oriented-affine-transform-of-cdf}
  \uses{def:oriented-affine-isomorphism, def:cdf}
  \lean{CumulativeDistributionFunction.affineTransform}
  \leanok
  The action of an orientation preserving affine isomorphism $A \in \OriAffR$
  on a cumulative distribution function $F$ is defined so
  that $A.F \colon \R \to \R$ is given by $(A.F)(x) = F(A^{-1}(x))$.
  Then $A.F$ is also a c.d.f.
\end{definition}

\begin{lemma}
  \label{lem:oriented-affine-action-on-cdf}
  \uses{def:oriented-affine-transform-of-cdf}
  \lean{CumulativeDistributionFunction.instMulActionAffineIncrEquiv}
  \leanok
  The actions of orientation preserving affine isomorphisms
  on a cumulative distribution functions is a group action, i.e.,
  $1.F = F$ and $(AB).F = A.(B.F)$ for any c.d.f. $F$ and any $A,B \in \OriAffR$.
\end{lemma}
\begin{proof}
  \uses{}
  % \leanok
  Direct calculations.
\end{proof}

\begin{lemma}
  \label{lem:degenerate-cdf-transform}
  \uses{lem:oriented-affine-action-on-cdf, def:degenerate-cdf}
  \lean{CumulativeDistributionFunction.affine_isDegenerate_iff}
  \leanok
  Let $F$ be a cumulative distribution function
  and $A \in \OriAffR$ an orientation preserving affine isomorphism.
  Then $A.F$ is degenerate if and only if $F$ is degenerate.
\end{lemma}
\begin{proof}
  \uses{}
  \leanok
  Straightforward from the definitions.
\end{proof}

\begin{lemma}
  \label{lem:cdf-continuity-pt-transform}
  \uses{lem:oriented-affine-action-on-cdf}
  \lean{CumulativeDistributionFunction.affine_continuousAt_of_continuousAt}
  \leanok
  Let $F$ be a cumulative distribution function,
  and $A \in \OriAffR$ an orientation preserving affine isomorphism.
  If a point $x \in \R$ is a continuity point of $F$, then
  the point $A(x) \in \R$ is a continuity point of $A.F$.
\end{lemma}
\begin{proof}
  \uses{}
  % \leanok
  Straightforward.
\end{proof}


% \begin{lemma}
% \label{lem:cdf-affine-group-action}
% \uses{def:cdf}
% Suppose $a > 0$ and $b \in \R$. Then for any c.d.f. $F$, also
% the function $x \mapsto F(a x + b)$ is a c.d.f.
% \end{lemma}
% \begin{proof}
% \ldots
% \end{proof}
%
% \begin{definition}
% \label{def:cdf}
% %\lean{CumulativeDistributionFunction}
% %\uses{def:degenerate-cdf}
% \uses{def:cdf}
% A function $F \colon \R \to \R$ is a
% \emph{cumulative distribution function} (c.d.f.) if
% \begin{enumerate}
% \item[(i)] $x \mapsto F(x)$ is increasing;
% \item[(ii)] $x \mapsto F(x)$ is right-continuous;
% \item[(iii)] $\lim_{x \to -\infty} F(x) = 0$ and $\lim_{x \to +\infty} F(x) = 1$.
% \end{enumerate}
% \end{definition}
%

\section{Miscellaneous results on cumulative distribution functions}

\begin{lemma}[Continuity points of c.d.f.s are those which carry no point mass]
  \label{lem:cdf-continuity-pt-iff-measure-singleton}
  \uses{def:cdf}
  \lean{CumulativeDistributionFunction.continuousAt_iff}
  \leanok
  Let $F$ be cumulative distribution function of a
  probability measure $\PRmeas$ on $\bR$. A point $x \in \bR$ is
  a continuity point of $F$ if and only if $\PRmeas[\set{x}] = 0$.
\end{lemma}
\begin{proof}
  % \uses{}
  % \leanok
  A c.d.f. is always continuous from the right.

  Continuity of $F$ from the left at $x$ means that for any
  sequence $(x_n)_{n \in \bN}$ increasing to $x$
  (i.e., $x_n \le x_{n+1} < x$ for all $n \in \bN$)
  \begin{align*}
    F(x_n) \to F(x) ,
  \end{align*}
  or equivalently in terms of measures
  \begin{align*}
    \PRmeas \big[ (-\infty,x_n] \big] \to \PRmeas \big[ (-\infty,x] \big] .
  \end{align*}
  But by monotone convergence of measures, we always have
  \begin{align*}
    \PRmeas \big[ (-\infty,x_n] \big]
    \to \; & \PRmeas \big[ (-\infty,x) \big] \\
    = \; & \PRmeas \big[ (-\infty,x] \big] - \PRmeas[\set{x}] .
  \end{align*}
  A comparison of these conditions shows that $F$ is also continuous from the left
  at $x$ if and only if $\PRmeas[\set{x}] = 0$.
\end{proof}

\begin{lemma}[A pair of nontrivial continuity points of nondegenerate c.d.f.]
  \label{lem:exists-two-nontrivial-continuity-pts-cdf}
  \uses{def:degenerate-cdf}
  \lean{CumulativeDistributionFunction.exists₂_continuousAt_of_not_isDegenerate}
  \leanok
  Let $G$ be a nondegenerate c.d.f. Then there exists
  continuity points $x_1 < x_2$ of $G$ such that
  $0 < G(x_1) \le G(x_2) < 1$.
\end{lemma}
\begin{proof}
  % \uses{}
  % \leanok
  Since $G$ is nondegenerate, there exists some $x_0 \in \bR$ such that
  $0 < G(x_0) < 1$. Since $G$ is continuous from the right, for some
  small $\delta > 0$ we have that $0 < G(x) < 1$ for
  all $x \in [x_0,x_0+\delta)$. Since the continuity points of $G$
  are dense, from any nonempty open interval we may pick a continuity point.
  First pick a continuity point $x_1 \in (x_0,x_0+\delta)$,
  and then pick another continuity point $x_2 \in (x_1,x_0+\delta)$.
\end{proof}


\begin{lemma}[Equality of c.d.f.s on a dense set suffices]
  \label{lem:cdf-equal-on-dense}
  \uses{def:cdf}
  \lean{CumulativeDistributionFunction.eq_of_forall_dense_eq}
  \leanok
  Suppose that $F,G$ are two c.d.f.s and $S \subseteq \bR$ is a dense
  subset of the real line. If $F(\xi) = G(\xi)$ for all $\xi \in S$,
  then we have $F = G$.
\end{lemma}
\begin{proof}
  % \uses{}
  % \leanok
  We must prove that for any $x \in \bR$ we have $F(x) = G(x)$.
  But by right-continuity of c.d.f.s, density of $S$, and coincidence of
  $F$ and $G$ on $S$, we have
  \begin{align*}
      F(x)
    = \; & \lim_{\xi \to x^+ \text{ along $S$}} F(\xi) \\
    = \; & \lim_{\xi \to x^+ \text{ along $S$}} G(\xi)
    = \, G(x) .
  \end{align*}
\end{proof}



\section{Topology on orientation-preserving affine isomorphisms}

\begin{definition}[]
  \label{def:affine-transform-topology}
  \uses{def:oriented-affine-isomorphism}
  \lean{instTopologicalSpaceAffineIncrEquiv}
  \leanok
  We equip the space $\OriAffR$ of orientation-preserving affine isomorphisms
  with the topology of pointwise convergence, i.e., with the
  coarsest topology which makes the evaluations $A \mapsto A(x)$ continuous
  for all $x \in \bR$.
\end{definition}

\begin{lemma}[The coefficients of affine map depend continuously on the map]
  \label{lem:affine-coefficients-continuous}
  \uses{def:oriented-affine-isomorphism}
  \lean{AffineIncrEquiv.continuous_coefs_fst, AffineIncrEquiv.continuous_coefs_snd}
  \leanok
  The coefficients $a$ and $b$ of an orientation-preserving affine isomorphism
  $A(x) = a x + b$ depend continuously on $A$.
\end{lemma}
\begin{proof}
  % \uses{}
  % \leanok
  We may first write $a = A(1) - A(0)$ and $b = A(0)$.
  These depend continuously on $A$,
  since the evaluations $A(1)$ and $A(0)$ do.
\end{proof}

\begin{lemma}[Metrizability of the topology on oriented affine isomorphisms]
  \label{lem:affine-metrizable}
  \uses{def:affine-transform-topology}
  \lean{AffineIncrEquiv.homeomorph_prod, AffineIncrEquiv.metrizableSpace}
  \leanok
  The topology of pointwise convergence makes $\OriAffR$
  homeomorphic to $\bR^2$, and in particular
  metrizable.
\end{lemma}
\begin{proof}
  \uses{lem:affine-coefficients-continuous}
  % \leanok
  The essential claim is that the function
  \begin{align*}
    \mathrm{cfs} \colon \OriAffR \to (0,+\infty) \times \bR
  \end{align*}
  obtained by mapping $A(x) = a x + b$ to its coefficients $(a,b)$ is a homeomorphism.
  (The homeomorphism to $\bR^2 = \bR \times \bR$ follows by combining with the
  homeomorphism $(a,b) \mapsto (\log a, b)$.)

  The continuity of $\mathrm{cfs}$ follows from Lemma~\ref{lem:affine-coefficients-continuous}.
  For the continuity of the inverse, we must only check that for any $x \in \bR$,
  its composition with the point evaluation at $x$ is continuous. But the composition
  is $(a,b) \mapsto a x + b$, and the continuity is clear.
\end{proof}

\begin{lemma}[Inversion of orientation preserving affine isomorphisms is continuous]
  \label{lem:affine-inversion-continuous}
  \uses{def:oriented-affine-isomorphism}
  \lean{AffineIncrEquiv.continuous_inv, AffineIncrEquiv.invHomeomorph}
  \leanok
  The map $A \mapsto A^{-1}$ is continuous on $\OriAffR$.
  In particular, inversion defines a homeomorphism of $\OriAffR$ to itself.
\end{lemma}
\begin{proof}
  \uses{lem:affine-coefficients-continuous}
  % \leanok
  Calculate and use Lemma~\ref{lem:affine-coefficients-continuous}.
\end{proof}

\begin{lemma}[The action of oriented affine transforms on c.d.f.s is continuous]
  \label{lem:action-on-cdf-continuous}
  \uses{lem:oriented-affine-action-on-cdf}
  \lean{CumulativeDistributionFunction.continuous_mulAction}
  \leanok
  The action $A . F$ of $A \in \OriAffR$ on a c.d.f $F$ depends
  jointly continuously on $A$ and $F$.

  \emph{(The topology on c.d.f.s is the topology of convergence in distribution, i.e.,
  convergence at all continuity points of the limit cdf.)}
\end{lemma}
\begin{proof}
  \uses{lem:affine-metrizable, lem:affine-inversion-continuous}
  % \leanok
  The spaces are metrizable, so it suffices to check sequential continuity.

  Suppose that $A_n \to B$ (oriented affine isomophisms)
  and $F_n \tendsinlaw G$ (c.d.f.s) as $n \to \infty$.
  Let $x \in \bR$ be a continuity point of $B.G$.
  Let $\eps > 0$. Note that $B^{-1}(x)$ is a continuity point of $G$.
  Then there exists a $\delta > 0$ such that $|G(y) - G\big(B^{-1}(x)\big)| < \frac{\eps}{2}$
  when $|y-B^{-1}(x)|<\delta$.

  By density of continuity points of $G$, pick continuity points
  $y_-,y_+$ such that
  \begin{align*}
    B^{-1}(x)-\delta \; < \; y_- \; < \; B^{-1}(x) \; < \; y_+ \; < \; B^{-1}(x)+\delta .
  \end{align*}
  Since $F_n \tendsinlaw G$, we have that
  $F_n(y_\pm) \to G(y_\pm)$ as $n \to \infty$,
  and in particular there exists some $N$ such that for $n \ge N$
  we have $\big| F_n(y_\pm) - G(y_\pm) \big| < \frac{\eps}{4}$ for both $y_\pm$.

  By Lemma~\ref{lem:affine-inversion-continuous}
  and $A_n \to B$, we get that $A_n^{-1} \to B^{-1}$, and in particular
  there exists some $N'$ such that for $n \ge N'$ we have
  \begin{align*}
    y_- < A_n^{-1}(x) < y_+ .
  \end{align*}

  For $n \ge \max \set{N,N'}$, we then have
  \begin{align*}
    (A_n.F_n)(x) \;
    = \; & F_n \big( A_n^{-1}(x) \big) \\
    \le \; & F_n (y_+) \\
    < \; & G(y_+) + \frac{\eps}{4} \\
    \le \; & G \big( B^{-1}(x) \big) + \frac{\eps}{2} + \frac{\eps}{4} \\
    = \; & (B.G) (x) + \frac{3\eps}{4} \\
    < \; & (B.G) (x) + \eps
  \end{align*}
  and similarly
  \begin{align*}
    (A_n.F_n)(x) \;
    = \; & F_n \big( A_n^{-1}(x) \big) \\
    \ge \; & F_n (y_-) \\
    > \; & G(y_-) - \frac{\eps}{4} \\
    \ge \; & G \big( B^{-1}(x) \big) - \frac{\eps}{2} - \frac{\eps}{4} \\
    = \; & (B.G) (x) - \frac{3\eps}{4} \\
    > \; & (B.G) (x) + \eps ,
  \end{align*}
  which together yield $\big| (A_n.F_n)(x) - (B.G)(x) \big| < \eps$.

  Since $x$ was an arbitrary continuity point of $B.G$ and
  $\eps > 0$ was arbitrary, this proves that $A_n.F \tendsinlaw B.G$.
\end{proof}

% lem:oriented-affine-action-on-cdf
