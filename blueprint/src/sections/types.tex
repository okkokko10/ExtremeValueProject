\chapter{Types}

\begin{definition}[Type (of distribution on $\bR$)]
  \label{def:cdf-type}
  \uses{def:cdf, lem:oriented-affine-action-on-cdf}
  %\lean{lcinv}
  %\leanok
  Two c.d.f.s $F, G$ are said to be of the same type, if
  there exists an order-preserving affine isomorphism $A \in \OriAffR$
  such that $G = A . F$.

  Being of the same type is an equivalence relation, and
  the equivalence classes are called types (of distributions on $\bR$).
\end{definition}


\section{Convergence to types}

The notion of convergence of cumulative distribution functions
considered here is always taken to be pointwise convergence
on the set of continuity points of the limit c.d.f.
By Theorem~\ref{thm:convergence-in-distribution-with-cdf} and
Lemma~\ref{lem:cdf-convergence-from-convergence-in-distribution},
this corresponds to convergence in distribution (weak convergence
of probability measures). Below, when we write $F_n \tendsinlaw F$
for c.d.f.s $F_n$, $n \in \bN$, and $F$, this is always what we
mean.

For affine maps $A \colon \bR \to \bR$, we use the topology
of pointwise convergence of functions. Equivalently, convergence of affine maps
means the convergence of the coefficients $a, b \in \bR$ in the expression
$x \mapsto a x + b$ of the affine functions
(so $A_n \to A$ if and only if the functions are of the form
$A_n(x) = a_n x + b_n$ and $A(x) = a x + b$, and $a_n \to a$ and $b_n \to b$).

\begin{lemma}[Unique affine relation among two nondegenerate c.d.f.s]
  \label{lem:unique-affine-relation-cdf}
  \uses{def:cdf, def:degenerate-cdf, lem:oriented-affine-action-on-cdf}
  \lean{CumulativeDistributionFunction.unique_orientationPreservingAffineEquiv_smul_eq_not_isDegenerate}
  \leanok
  Let $F, G$ be two c.d.f.s of the same type, and $A \in \OriAffR$
  an affine isomorphism such that $G = A.F$. If $F$ is nondegenerate,
  then $A$ is the only element of $\OriAffR$ for which the
  relation $G = A.F$ holds.
\end{lemma}
\begin{proof}
  % \uses{}
  % \leanok
  Since $F$ is nondegenerate, we can find two different points $x_1 < x_2$
  such that $0 < F(x_1) < F(x_2) \le 1$. By right continuity of $F$, we can
  assume these points to be taken minimal with the given values, i.e.,
  $x_j = \inf \big\{ x \in \bR \; \big| \; F(x) = F(x_j) \big\}$ for
  $j=1,2$.

  The assumption $G = A.F$ means $G(x) = F \big( A^{-1}(x) \big)$ for all $x \in \bR$.
  Therefore $G \big( A(x_1) \big) = F(x_1) < F(x_2) = G \big( A(x_2) \big)$.
  We also get $A(x_j) = \inf \big\{ y \in \bR \; \big| \; G(y) = F(x_j) \big\}$ for
  $j=1,2$ by strict monotonicity and bijectivity of $A$
  (if, for example, there would exist a $y'_2 < A(x_2)$ such that $G(y'_2) = F(x_2)$,
  then the point $x'_2 = A^{-1}(y'_2) < x_2$ would be such that
  $F\big(A^{-1}(y'_2)\big) = G(y'_2) = F(x_2)$, contradicting the minimality of $x_2$).

  If $\widetilde{A} \in \OriAffR$ is also such that $G = \widetilde{A}.F$, then
  the same holds for it:
  $\widetilde{A}(x_j) = \inf \big\{ y \in \bR \; \big| \; G(y) = F(x_j) \big\}$
  for $j=1,2$. We conclude that
  \begin{align*}
    \widetilde{A}(x_1) \, = \; &
        \inf \big\{ y \in \bR \; \big| \; G(y) = F(x_1) \big\} \; = \, A(x_1) \\
    \widetilde{A}(x_2) \, = \; &
        \inf \big\{ y \in \bR \; \big| \; G(y) = F(x_2) \big\} \; = \, A(x_2) .
  \end{align*}
  But an affine map of $\bR$ is determined by its values at two distinct points:
  from $a x_1 + b = y_1$ and $a x_2 + b = y_2$ with $x_1 \ne x_2$ one can solve $a, b$.
  Therefore we must have $\widetilde{A} = A$.
\end{proof}

\begin{lemma}[Degeneration by shrinking affine transformations]
  \label{lem:degenerate-shrinking-limit}
  \uses{def:cdf-degenerate, lem:oriented-affine-action-on-cdf}
  \lean{CumulativeDistributionFunction.isDegenerate_of_tendsto_shrinking}
  \leanok
  Let $(F_n)_{n \in \bN}$ be a sequence of c.d.f.s which converges to a c.d.f. $G$,
  $F_n \tendsinlaw G$. Consider affine transformations of the form $A_n(x) = a_n x + b_n$,
  with $a_n > 0$ and $b_n \in \bR$, such that $a_n \to 0$ and $b_n \to \beta \in \bR$
  as $n \to \infty$. Then $A_n . F_n \tendsinlaw \widetilde{G}$, where
  $\widetilde{G}$ is the degenerate c.d.f. of the delta mass at $\beta$.
\end{lemma}
\begin{proof}
  % \uses{}
  % \leanok
  It suffices to prove that for any $x < \beta$ we have $\widetilde{G}(x) = 0$
  and for any $x > \beta$ we have $\widetilde{G}(x) = 1$.

  Let us focus on the latter, so let $x > \beta$. Let also $\eps > 0$; we will prove
  that $\widetilde{G}(x) > 1 - \eps$, and the claim will follow.

  By density of continuity points of $\widetilde{G}$, we can choose $x'$ such that
  $\beta < x' < x$ and $\widetilde{G}$ is continuous at $x'$. Then the assumed
  convergence $A_n . F_n \tendsinlaw \widetilde{G}$ implies that
  $(A_n . F_n)(x') \to \widetilde{G}(x')$. Since $\widetilde{G}(x) \ge \widetilde{G}(x')$,
  it suffices to prove that $\widetilde{G}(x') > 1 - \eps$.

  Since $G$ is a c.d.f., we can choose a continuity point $z$ of $G$
  large enough so that $G(z) > 1 - \eps$.
  Then by the assumed convergence $F_n \tendsinlaw G$, we have
  $F_n(z) \to G(z)$. By definition we have
  \begin{align*}
    (A_n . F_n)(A_n(z)) \; = \; F_n \big( A_n^{-1} (A_n(z)) \big) \; = \; F_n(z)
      \; \longrightarrow \; G(z) . %\; > \;  1-\eps .
  \end{align*}
  Note that $A_n(z) = a_n z + b_n \to \beta$ as $n \to \infty$ by the assumptions
  $a_n \to 1$ and $b_n \to \beta$. In particular, for $n$ large enough,
  we have $A_n(z) < x'$.
  Therefore, for $n$ large enough
  \begin{align*}
    F_n(z) \; = \; (A_n . F_n) \big( A_n(z) \big) \; \le \; (A_n . F_n)(x') .
  \end{align*}
  The LHS tends to $G(z)$ and the RHS tends to $\widetilde{G}(x')$,
  showing
  \begin{align*}
    1 - \eps \; < \; G(z) \; \le \; \widetilde{G}(x') \; \le \; \widetilde{G}(x) .
  \end{align*}
  This concludes the proof that $\widetilde{G}(x) = 1$ for all $x > \beta$.

  The proof that $\widetilde{G}(x) = 0$ for all $x < \beta$ is similar.
\end{proof}

\begin{lemma}[Impossibility of expanding affine transformations]
  \label{lem:impossible-expanding-limit}
  \uses{def:cdf, lem:oriented-affine-action-on-cdf}
  \lean{CumulativeDistributionFunction.not_tendsto_cdf_of_expanding_of_tendsto_not_isDegenerate}
  \leanok
  Let $(F_n)_{n \in \bN}$ be a sequence of c.d.f.s which converges to a nondegenerate
  c.d.f. $G$, $F_n \tendsinlaw G$. Consider affine transformations of the form $A_n(x) = a_n x + b_n$,
  with $a_n > 0$ and $b_n \in \bR$, such that $a_n \to +\infty$
  as $n \to \infty$. Then $A_n . F_n$ cannot converge to any c.d.f.
\end{lemma}
\begin{proof}
  \uses{lem:exists-two-nontrivial-continuity-pts-cdf}
  % \leanok
  Since $G$ is assumed nondegenerate, by Lemma~\ref{lem:exists-two-nontrivial-continuity-pts-cdf}
  we can pick two continuity points $x_1 < x_2$
  of $G$ such that $0 < G(x_1) \le G(x_2) < 1$. Then from the assumption
  $F_n \tendsinlaw G$ we get $F_n(x_1) \to G(x_1)$ and $F_n(x_2) \to G(x_2)$.

  Assume, by way of contradiction, that we have convergence
  $A_n . F \tendsinlaw \widetilde{G}$ to some c.d.f. $\widetilde{G}$.

  We claim that then $\big(A_n(x_1)\big)_{n \in \bN}$ is bounded from below and
  $\big(A_n(x_2)\big)_{n \in \bN}$ is bounded from above.
  Since
  \begin{align*}
    a_n = \frac{A_n(x_2) - A_n(x_1)}{x_2 - x_1} ,
  \end{align*}
  this will show that $(a_n)_{n \in \bN}$ is bounded from above, contradicting
  the assumption $a_n \to +\infty$, and finishing the proof.

  To show that $\big(A_n(x_1)\big)_{n \in \bN}$ is bounded from below,
  choose a continuity point $z$ of $\widetilde{G}$ such that
  $\widetilde{G}(z) < G(x_1)$. Then the assumed convergence
  $A_n . F \tendsinlaw \widetilde{G}$ implies
  $(A_n . F)(z) \to \widetilde{G}(z)$.
  On the other hand, if $\big(A_n(x_1)\big)_{n \in \bN}$ is not
  bounded from below, then along some subsequence $(n_k)_{k \in \bN}$ of indices
  we have $A_{n_k}(x_1) < z$, and for those indices we then have
  \begin{align*}
    F_{n_k}(x_1) \; = \; (A_{n_k} . F_{n_k}) \big( A_{n_k}(x_1) \big)
      \; \le \; (A_{n_k} . F_{n_k})(z) .
  \end{align*}
  The LHS tends to $G(x_1)$ as $k \to \infty$, whereas the RHS tends
  to $\widetilde{G}(z)$. We get $G(x_1) \le \widetilde{G}(z)$, contradicting
  the choice of $z$. This shows that $\big(A_n(x_1)\big)_{n \in \bN}$ must
  in fact be bounded from below.

  The proof that $\big(A_n(x_2)\big)_{n \in \bN}$ must be bounded from above
  is similar.
\end{proof}

\begin{theorem}[Convergence to types]
  \label{thm:convergence-to-types}
  \uses{def:degenerate-cdf, lem:oriented-affine-action-on-cdf, def:cdf-type}
  % \lean{}
  % \leanok
  Suppose that $(F_n)_{n \in \bN}$ is a sequence of
  c.d.f.s which converges to a nondegenerate c.d.f. $G$, i.e.,
  $F_n \tendsinlaw G$ as $n \to \infty$.
  Let $(A_n)_{n \in \bN}$ be a sequence of oriented
  affine isomorphisms of $\bR$, $A_n \in \OriAffR$
  such that $A_n.F_n \tendsinlaw \widetilde{G}$ for some c.d.f.
  $\widetilde{G}$.

  If we write $A_n(x) = a_n x + b_n$, then $(a_n)_{n \in \bN}$
  and $(b_n)_{n \in \bN}$ are bounded sequences.

  If $\widetilde{G}$ is nondegenerate, then
  $A_n \to A \in \OriAffR$ and $A.G = \widetilde{G}$.
  In particular $G$ and $\widetilde{G}$ are of the same type.
  Moreover, $A$ is the unique affine transformation for which
  the equality $A.G = \widetilde{G}$ holds.
\end{theorem}
\begin{proof}
  \uses{lem:impossible-expanding-limit, lem:degenerate-shrinking-limit,
    lem:unique-affine-relation-cdf, lem:action-on-cdf-continuous}
  % \leanok
  Let us first argue that $(a_n)_{n \in \bN}$ are bounded.
  If not, then by passing to a subsequence, we have $a_{n_k} \to +\infty$.
  But since $F_{n_k} \tendsinlaw G$ and $G$ is nondegenerate,
  it contradicts Lemma~\ref{lem:impossible-expanding-limit}
  to have $A_{n_k} . F_{n_k} \tendsinlaw \widetilde{G}$.
  Therefore $(a_n)_{n \in \bN}$ must be bounded: there exists
  some $M>0$ such that $a_n \le M$ for all $n \in \bN$.

  Let us then argue that $(b_n)_{n \in \bN}$ is bounded.
  If not, then we can extract a subsequence such
  that either $b_{n_k} \to -\infty$ or $b_{n_k} \to +\infty$.
  Let us prove the impossibility of the second one, the first is similar.
  So assume that $b_{n_k} \to +\infty$.
  Since $G$ is nondegenerate, we may pick a continuity point $x_0$
  of $G$ such that $0 < G(x_0) < 1$. Then we have $F_n(x_0) \to G(x_0)$
  by the assumption $F_n \tendsinlaw G$.
  We may also pick a continuity point $z$ of $\widetilde{G}$
  such that $\widetilde{G}(z) > G(x_0)$. Then we have
  $(A_n.F_n)(z) \to \widetilde{G}(z)$
  by the assumption $A_n.F_n \tendsinlaw \widetilde{G}$.
  But $A_{n_k}(x_0) = a_{n_k} x_0 + b_{n_k} \to +\infty$, since $0 < a_{n_k} \le M$
  and $b_{n_k} \to +\infty$. Therefore we have for all large enough $k$
  that $A_{n_k}(x_0) > z$. And then
  \begin{align*}
    (A_{n_k}.F_{n_k})(z) \; \le \; (A_{n_k}.F_{n_k})\big(A_{n_k}(x_0)\big) \; = \; F_{n_k}(x_0) .
  \end{align*}
  The LHS tends to $\widetilde{G}(z)$ as $k \to \infty$, and the RHS tends to $G(x_0)$.
  Therefore we get $\widetilde{G}(z) \le G(x_0)$, contradicting the choice of $z$.
  This shows that we cannot have $b_{n_k} \to +\infty$. Similarly one proves that
  we cannot have $b_{n_k} \to -\infty$. We conclude that $(b_n)_{n \in \bN}$
  is indeed bounded.

  From now on, suppose furthermore that also $\widetilde{G}$ is nondegenerate.
  We claim that then $(a_n)_{n \in \bN}$ is bounded away from $0$:
  for some $\eps > 0$ we have $a_n \ge \eps$ for all $n \in \bN$.
  If not, then we could extract a subsequence such that $a_{n_k} \to 0$
  and also $b_n \to \beta$ (since $b_n$ is bounded).
  But since $F_{n_k} \tendsinlaw G$ and $G$ is nondegenerate,
  from Lemma~\ref{lem:degenerate-shrinking-limit}
  we would get $A_{n_k} . F_{n_k} \tendsinlaw \widetilde{G}_0$,
  where $\widetilde{G}_0$ is degenerate. But by assumption
  $A_{n_k} . F_{n_k} \tendsinlaw \widetilde{G}$, where
  $\widetilde{G}$ is nondegenerate; this is impossible by uniqueness
  of limits for convergence in distribution. Therefore
  $(a_n)_{n \in \bN}$ must indeed be bounded away from $0$.

  Note that since $(a_n)_{n \in \bN}$ is bounded away from $0$ and $+\infty$,
  and $(b_n)_{n \in \bN}$ is bounded, we can extract a subsequence such that
  $a_{n_k} \to \alpha \in (0,+\infty)$ and $b_{n_k} \to \beta \in \bR$.
  By assumption, we have $A_{n_k} . F_{n_k} \tendsinlaw \widetilde{G}$.
  But since $A_{n_k} \to A$ where $A(x) = \alpha x + \beta$
  and we have also assumed $F_{n_k} \tendsinlaw G$, this implies by
  continuity (Lemma~\ref{lem:action-on-cdf-continuous}) that $A_{n_k} . F_{n_k} \tendsinlaw A . G$.
  By uniqueness of limits, we get $A . G = \widetilde{G}$.

  To prove that in fact $A_n \to A$, not just along a subsequence, note the
  following. From any subsequence $A_{n_k}$, we can extract a further
  convergent subsequence of values of $a_{n_{k_\ell}}$ and $b_{n_{k_\ell}}$
  values as above, with limits $\alpha' \in (0,+\infty)$
  and $\beta' \in \bR$. The same argument as above shows that
  $A' . G = \widetilde{G}$ where $A'(x) = \alpha' x + \beta'$.
  Lemma~\ref{lem:unique-affine-relation-cdf} then says that we must have
  $A' = A$, i.e., $\alpha' = \alpha$ and $\beta' = \beta$. Since any
  subsequence has a convergent further subsequence with the same limit,
  the entire sequence must converge, $A_n \to A$.
  The proof is complete.
\end{proof}

\begin{theorem}[Convergence to types again]
  \label{thm:convergence-to-types-different-affine}
  \uses{def:cdf, lem:oriented-affine-action-on-cdf, def:cdf-type}
  % \lean{} % also `AffineIncrEquiv.inv_mul_eq_mkOfCoefs`
  % \leanok
  Let $(A_n)_{n \in \bN}$ and
  $(\widetilde{A}_n)_{n \in \bN}$ be two sequences of oriented
  affine isomorphisms of $\bR$, $A_n, \widetilde{A}_n \in \OriAffR$.
  Write $A_n(x) = a_n x + b_n$
  and $\widetilde{A}_n(x) = \tilde{a}_n x + \tilde{b}_n$,
  and for the inverses $A_n^{-1}(x) = c_n x + d_n$
  and $\widetilde{A}^{-1}_n(x) = \tilde{c}_n x + \tilde{d}_n$.

  Let $(F_n)_{n \in \bN}$ be a sequence of c.d.f.s
  such that $A_n.F_n \tendsinlaw G$, with
  $G$ a nondegenerate c.d.f.

  Then the convergence of also $\widetilde{A}_n.F_n \tendsinlaw G$
  holds if and only if the coefficients of the affine maps
  satisfy the relations
  \begin{align*}
    \frac{\tilde{a}_n}{a_n} \to 1
    \quad \text{ and } \quad
    \frac{a_n \tilde{b}_n - \tilde{a}_n b_n}{a_n} \to 0,
  \end{align*}
  or equivalently,
  \begin{align*}
    \frac{\tilde{c}_n}{c_n} \to 1
    \quad \text{ and } \quad
    \frac{\tilde{d}_n - d_n}{c_n} \to 0 .
  \end{align*}
\end{theorem}
\begin{proof}
  \uses{thm:convergence-to-types}
  % \leanok
  \emph{(This is actually just a special case of what is stated as
  Corollary~\ref{cor:convergence-to-types-different-limit} below.
  The better organization is to prove that corollary
  directly, and obtain this lemma as a special case.)}

  We will apply the convergence to types with the reference sequence
  $(A_n.F_n)_{n \in \bN}$, which by assumption tends to a
  nondegenerate $G$.

  To express the other sequence in terms of the reference sequence, we write
  \begin{align*}
    \widetilde{A}_n.F_n \; = \; (\widetilde{A}_n A_n^{-1}).(A_n.F_n) .
  \end{align*}
  By assumption this also tends to $G$.

  Theorem~\ref{thm:convergence-to-types} applies, and guarantees convergence
  $\widetilde{A}_n A_n^{-1} \to A$ to some $A \in \OriAffR$,
  and it also implies $A.G = G$ (note that the other limit is also $G$
  by our assumptions).
  However, the unique $A$ for which we have $A.G = G$ is $A = \mathrm{id}$.
  We therefore get $\widetilde{A}_n A_n^{-1} \to \id$.
  To explicitly see the coefficients, note that
  \begin{align*}
    A_n^{-1}(x) \; = \; & a_n^{-1} (x - b_n)
  \end{align*}
  and
  \begin{align*}
    \widetilde{A}_n \big( A_n^{-1}(x) \big)
    \; = \; & \tilde{a}_n \big(a_n^{-1} (x - b_n)\big) + \tilde{b}_n \\
    \; = \; & \tilde{a}_n a_n^{-1} x - \tilde{a}_n a_n^{-1} b_n + \tilde{b}_n .
  \end{align*}
  The convergence $\widetilde{A}_n A_n^{-1} \to \id$ is equivalent to
  the convergence of the coefficients,
  \begin{align*}
    \tilde{a}_n a_n^{-1} \; \longrightarrow \; & 1 \\
    - \tilde{a}_n a_n^{-1} b_n + \tilde{b}_n \; \longrightarrow \; & 0 .
  \end{align*}
  The second one can be rewritten as
  \begin{align*}
    \frac{a_n \tilde{b}_n - \tilde{a}_n b_n}{a_n} \; \longrightarrow \; & 0 .
  \end{align*}
\end{proof}

\begin{corollary}[Convergence to types with different limits]
  \label{cor:convergence-to-types-different-limit}
  \uses{def:cdf, lem:oriented-affine-action-on-cdf, def:cdf-type}
  % \lean{} % also `AffineIncrEquiv.inv_mul_eq_mkOfCoefs`
  % \leanok
  Let $(A_n)_{n \in \bN}$ and
  $(\widetilde{A}_n)_{n \in \bN}$ be two sequences of oriented
  affine isomorphisms of $\bR$, $A_n, \widetilde{A}_n \in \OriAffR$.
  Write $A_n(x) = a_n x + b_n$
  and $\widetilde{A}_n(x) = \tilde{a}_n x + \tilde{b}_n$,
  and for the inverses $A_n^{-1}(x) = c_n x + d_n$
  and $\widetilde{A}^{-1}_n(x) = \tilde{c}_n x + \tilde{d}_n$.

  Let $(F_n)_{n \in \bN}$ be a sequence of c.d.f.s
  such that $A_n.F_n \tendsinlaw G$ and
  $\widetilde{A}_n.F_n \tendsinlaw \widetilde{G}$, with
  $G$ and $\widetilde{G}$ nondegenerate c.d.f.s.
  Then for some $\alpha>0$ and $\beta \in \bR$ we have
  \begin{align*}
    \frac{\tilde{a}_n}{a_n} \to \alpha
    \quad \text{ and } \quad
    \frac{a_n \tilde{b}_n - \tilde{a}_n b_n}{a_n} \to \beta,
  \end{align*}
  and we have
  \begin{align*}
    A.G = \widetilde{G}
    \qquad \text{ where } \; A(x) = \alpha x + \beta .
  \end{align*}
  Equivalently, with $\gamma = \alpha^{-1}$ and $\delta = -\alpha^{-1} \beta$
  so that $A^{-1}(x) = \gamma x + \delta$, we have
  \begin{align*}
    \frac{\tilde{c}_n}{c_n} \to \gamma
    \quad \text{ and } \quad
    \frac{\tilde{d}_n - d_n}{c_n} \to \delta .
  \end{align*}

  In particular, $G$ and $\widetilde{G}$ have the same type.
\end{corollary}
\begin{proof}
  \uses{thm:convergence-to-types-different-affine}
  % \leanok
  We will apply the convergence to types with the reference sequence
  $(A_n.F_n)_{n \in \bN}$, which by assumption tends to a
  nondegenerate $G$.

  To express the other sequence in terms of the reference sequence, we write
  \begin{align*}
    \widetilde{A}_n.F_n \; = \; (\widetilde{A}_n A_n^{-1}).(A_n.F_n) .
  \end{align*}
  By assumption this tends to a nondegenerate $\widetilde{G}$.

  Theorem~\ref{thm:convergence-to-types} applies, and guarantees convergence
  $\widetilde{A}_n A_n^{-1} \to A$ to some $A \in \OriAffR$,
  and it also implies $A.G = \widetilde{G}$.

  Write $A(x) = \alpha x + \beta$. To explicitly see the coefficients of
  $\widetilde{A}_n A_n^{-1}$, note that
  \begin{align*}
    A_n^{-1}(x) \; = \; & a_n^{-1} (x - b_n)
  \end{align*}
  and
  \begin{align*}
    \widetilde{A}_n \big( A_n^{-1}(x) \big)
    \; = \; & \tilde{a}_n \big(a_n^{-1} (x - b_n)\big) + \tilde{b}_n \\
    \; = \; & \tilde{a}_n a_n^{-1} x - \tilde{a}_n a_n^{-1} b_n + \tilde{b}_n .
  \end{align*}
  The convergence $\widetilde{A}_n A_n^{-1} \to A$ is equivalent to
  the convergence of the coefficients,
  \begin{align*}
    \tilde{a}_n a_n^{-1} \; \longrightarrow \; & \alpha \\
    - \tilde{a}_n a_n^{-1} b_n + \tilde{b}_n \; \longrightarrow \; & \beta .
  \end{align*}
  The second one can be rewritten as
  \begin{align*}
    \frac{a_n \tilde{b}_n - \tilde{a}_n b_n}{a_n} \; \longrightarrow \; & \beta .
  \end{align*}
\end{proof}

\begin{lemma}[A choice of normalizing constants for convergence to types]
  \label{lemma:convergence-to-types-normalization}
  \uses{def:cdf, lem:oriented-affine-action-on-cdf, def:cdf-type, def:lc-inverse}
  % \lean{}
  % \leanok
  (It is possible to choose normalization constants for the affine
  transformations using the left-continuous inverses of the c.d.f.s.
  TODO: Precise statement.)
%  Suppose that $(F_n)_{n \in \bN}$ is a sequence of c.d.f.s
%  and $(A_n)_{n \in \bN}$ %and $(\widetilde{A}_n)_{n \in \bN}$
%  be a sequence of oriented
%  affine isomorphisms of $\bR$, $A_n \in \OriAffR$,
%  such that $A_n.F_n \tendsinlaw G$ where
%  $G$ is a nondegenerate c.d.f.
%
%  (TODO: The following is not quite correct. Need to assume one chooses
%  continuity points of the left-continuous inverse.)
%
%  Then for any $0 < p_1 < p_2 < 1$ and $0 < p_0 < 1$, defining
%  \begin{align*}
%    \tilde{a}_n = \frac{1}{\lcinv{F}_n(p_2) - \lcinv{F}_n(p_1)}
%    \quad \text{ and } \quad
%    \tilde{b}_n = \frac{\lcinv{F}_n(p_0)}{\lcinv{F}_n(p_2) - \lcinv{F}_n(p_1)}
%  \end{align*}
%  we have that
%  \begin{align*}
%    \widetilde{A}_n.F_n \tendsinlaw \widetilde{G}
%    \qquad \text{ where } \; \widetilde{A}_n(x) = \tilde{a}_n x + \tilde{b}_n .
%  \end{align*}
\end{lemma}
\begin{proof}
  \uses{thm:convergence-to-types-different-affine}
  % \leanok
  \ldots
\end{proof}


\section{One-parameter subgroups of affine transformations}

\begin{definition}[Subgroup of translations]
  \label{def:translation-subgroup}
  \uses{def:oriented-affine-isomorphism}
  \lean{AffineIncrEquiv.homOfIndex₀, AffineIncrEquiv.subGroupOfIndex₀}
  % \leanok
  The mapping $s \mapsto A_s$ with
  \begin{align*}
    A_s(x) = x + s
  \end{align*}
  is a homomorphism $\bR \to \OriAffR$. The image of this
  homomorphism is the \emph{subgroup of translations} in $\OriAffR$.
\end{definition}

\begin{lemma}[Only translations have no fixed points]
  \label{lem:no-fixed-point-implies-translation}
  \uses{def:translation-subgroup}
  \lean{AffineIncrEquiv.mem_subGroupOfIndex₀_of_no_fixed_point}
  \leanok
  If $A \in \OriAffR$ has no fixed points (no $x \in \bR$ such that $A(x) = x$)
  then $A$ belongs to the subgroup of translations, i.e., $A(x) = x + s$
  for some $s \in \bR$ (in fact $s \ne 0$).
\end{lemma}
\begin{proof}
  % \uses{}
  % \leanok
  Let us prove this by contrapositive: that any element $A$ which is not a
  translation must have a fixed point.
  So assume that $A$ is not a translation, i.e.,
  $A(x) = a x + b$ with some $a \ne 1$. Then the fixed point equation
  $A(x) = x$ reads
  \begin{align*}
    a x + b = x  ,
  \end{align*}
  and it has a solution $x = \frac{-b}{a-1} \in \bR$, which then is a
  fixed point of $A$.
\end{proof}

\begin{lemma}[Conjugate of translation is translation]
  \label{lem:conjugate-translation}
  \uses{def:translation-subgroup}
  \lean{AffineIncrEquiv.conjugate_homOfIndex₀}
  % \leanok
  Let $A^{(\beta)}_s = x + \beta s$ for $s, \beta \in \bR$ as in
  Definition~\ref{def:translation-subgroup}. Let
  also $B \in \OriAffR$ be given by $B(x) = a x + b$. Then
  \begin{align*}
    B \, A^{(\beta)}_s \, B^{-1} = A^{(a \beta)}_{s} .
  \end{align*}
\end{lemma}
\begin{proof}
  % \uses{}
  % \leanok
  Calculate, for $x \in \bR$
  \begin{align*}
    (B A^{(\beta)}_s B^{-1})(x) \; = \; & (B A^{(\beta)}_s)\big( \frac{x-b}{a} \big) \\
    \; = \; & B\big( \frac{x-b}{a} + \beta s \big) \\
    \; = \; & a \big( \frac{x-b}{a} + \beta s \big) + b \\
    \; = \; & x - b + a \beta s + b \\
    \; = \; & x + a \beta s \; = \; A^{(a \beta)}_s(x).
  \end{align*}
\end{proof}

\begin{definition}[Subgroup fixing a point]
  \label{def:fixing-subgroup}
  \uses{def:oriented-affine-isomorphism}
  \lean{AffineIncrEquiv.homOfIndex, AffineIncrEquiv.subGroupOfIndex}
  % \leanok
  The mapping $s \mapsto A_s$ with
  \begin{align*}
    A_s(x) = e^{s} (x - c) + c
  \end{align*}
  is a homomorphism $\bR \to \OriAffR$. The image of this
  homomorphism is the \emph{subgroup fixing $c$} in $\OriAffR$.
\end{definition}

\begin{lemma}[Characterization of the subgroup fixing a point]
  \label{lem:fixing-subgroup-characterization}
  \uses{def:fixing-subgroup}
  \lean{AffineIncrEquiv.mem_subGroupOfIndex_iff_fixed_point}
  \leanok
  An orientation-preserving affine transformation $A \in \OriAffR$
  belongs to the subgroup fixing $c \in \bR$ if and only if $A(c) = c$.

  \emph{(Note that the subgroup is a priori defined as the image of a homomorphism,
  so the statement indeed requires a proof.)}
\end{lemma}
\begin{proof}
  % \uses{}
  % \leanok
  Suppose first that $A$ is an element of the said subgroup, i.e.,
  $A(x) = e^{s} (x - c) + c$ for some $s \in \bR$. Then clearly $A(c) = c$.

  Suppose then that $A(c) = c$. Write $A(x) = a x + b$ for $a>0$ and $b \in \bR$.
  Plug in $x=c$ in the assumed fixed point property to obtain
  \begin{align*}
    a c + b = c .
  \end{align*}
  The above can be solved to give $b = (1-a)c$.
  Also since $a>0$, we can write $a = e^s$ with $s \in \bR$.
  With these, the formula for $A$ simplifies to
  \begin{align*}
    A(x) = e^s x + c (1 - e^s) = e^s \big( x - c \big) + c .
  \end{align*}
  This shows $A = A_s$ as desired (with $A_s$ as in Definition~\ref{def:fixing-subgroup}).
\end{proof}

\begin{lemma}[Conjugate of fixing is fixing image]
  \label{lem:conjugate-fixing}
  \uses{def:fixing-subgroup}
  \lean{AffineIncrEquiv.conjugate_homOfIndex}
  % \leanok
  Let $A^{(\alpha;c)}_s = e^{\alpha s} (x - c) + c$ for $\alpha, c \in \bR$ as in
  Definition~\ref{def:fixing-subgroup}. Let
  also $B \in \OriAffR$ be given by $B(x) = a x + b$. Then
  \begin{align*}
    B \, A^{(\alpha;c)}_s \, B^{-1} = A^{(\alpha;B(c))}_{s} .
  \end{align*}
\end{lemma}
\begin{proof}
  % \uses{}
  % \leanok
  Calculate, for $x \in \bR$
  \begin{align*}
    (B A^{(\alpha;c)}_s B^{-1})(x) \; = \; & (B A^{(\alpha;c)}_s)\big( \frac{x-b}{a} \big) \\
    \; = \; & B\big( e^{\alpha s} (\frac{x-b}{a} - c) + c \big) \\
    \; = \; & a e^{\alpha s} (\frac{x-b}{a} - c) + a c + b \\
    \; = \; & e^{\alpha s} \big( x - (a c + b) \big) + (a c + b) \\
    \; = \; & e^{\alpha s} \big( x - B(c) \big) + B(c) \\
    \; = \; & A^{(\alpha;B(c))}_s (x) .
  \end{align*}
\end{proof}


\section{Self-similarity characterizations of the extreme value distributions}

\begin{lemma}[Continuous parameter extreme value limit relation]
  \label{lem:continuous-parameter-ev-limit}
  \uses{def:cdf, lem:oriented-affine-action-on-cdf}
  \lean{continuous_parameter_ev_limit_relation}
  \leanok
  Let $F$ be a c.d.f.

  \emph{(Note that below we use
  the sequence $(F^n)_{n \in \bN}$ of $n$th powers of a fixed c.d.f.,
  not a sequence of arbitrary c.d.f.s.
  Recall that the $n$th power $F^n$ is the c.d.f. of the maximum
  of $n$ independent random variables with the distribution $F$.)}

  Suppose that for a sequence $(A_n)_{n \in \bN}$ of oriented
  affine isomorphisms of $\bR$, $A_n \in \OriAffR$, we have
  \begin{align*}
    A_n.F^n \tendsinlaw G ,
  \end{align*}
  where $G$ is a c.d.f.

  Then, for any $t > 0$, denoting by $G^t$ the c.d.f. given by
  $G^t(x) = \big( G(x) \big)^t$, we have
  \begin{align*}
    A_{n}.F^{\lfloor n t \rfloor} \tendsinlaw G^t ,
  \end{align*}
  where, for $x \in \bR$, the floor notation $\lfloor x \rfloor$ stands
  for the greatest integer $k \in \bZ$ such that $k \le x$.
\end{lemma}
\begin{proof}
  % \uses{}
  % \leanok
  Let $t > 0$ and let $x \in \bR$ be a continuity point of $G$.
  For $n \in \bN$, calculate
  \begin{align*}
      \big((A_n . F) (x) \big)^{\lfloor n t \rfloor}
    %\; = \; \big(F(A_n^{-1}(x)) \big)^{\lfloor n t \rfloor} \\
    \; = \; & \Big( \big((A_n . F) (x) \big)^n \Big)^{\lfloor n t \rfloor / n} .
  \end{align*}
  By assumption, we have $\big((A_n . F) (x) \big)^n \to G(x)$ as $n \to \infty$.
  Also $\lfloor n t \rfloor / n \to t$ as $n \to \infty$.
  By (joint) continuity of the power function
  $(x,y) \mapsto x^y = \exp\big( y \, \log(x) \big)$,
  we get that the expression above tends to $G(x)^t$.

  Finally noting that the continuity points of $G^t$ are the same as the
  continuity points of $G$, the above in fact proves the asserted
  $A_{n}.F^{\lfloor n t \rfloor} \tendsinlaw G^t$.
\end{proof}

\begin{lemma}[Self-similarity of extreme value distributions]
  \label{lem:self-similarity-of-extreme-value-distributions}
  \uses{def:cdf, lem:oriented-affine-action-on-cdf, def:extr-val-distr}
  \lean{CumulativeDistributionFunction.IsExtremeValueDistr.self_similar}
  \leanok
  %Let $F$ be a c.d.f.
  %
  %\emph{(Note that below we use
  %the sequence $(F^n)_{n \in \bN}$ of $n$th powers of a fixed c.d.f.,
  %not a sequence of arbitrary c.d.f.s.
  %Recall that the $n$th power $F^n$ is the c.d.f. of the maximum
  %of $n$ independent random variables with the distribution $F$.)}
  %
  %Suppose that for a sequence $(A_n)_{n \in \bN}$ of oriented
  %affine isomorphisms of $\bR$, $A_n \in \OriAffR$, we have
  %\begin{align*}
  %  A_n.F^n \tendsinlaw G ,
  %\end{align*}
  %where $G$ is a c.d.f.
  Suppose that $G$ is an extreme-value distribution.
  Then there exists a family $(A_t)_{t > 0}$ of
  oriented affine isomorphisms of $\bR$, $A_t \in \OriAffR$,
  such that for any $t > 0$
  \begin{align*}
    G^t = A_t . G .
  \end{align*}

  Moreover, $t \mapsto A_t$ is a measurable homomorphism
  of multiplicative groups $(0,+\infty) \to \OriAffR$.
\end{lemma}
\begin{proof}
  \uses{lem:continuous-parameter-ev-limit, thm:convergence-to-types}
  % \leanok
  \ldots
\end{proof}

%\begin{lemma}[Continuous parameter limit relation]
%  \label{lem:continuous-parameter-ev-limit}
%  \uses{def:cdf, lem:oriented-affine-action-on-cdf}
%  % \lean{}
%  % \leanok
%  Let $F$ be a c.d.f.
%
%  Suppose that for a sequence $(A_n)_{n \in \bN}$ of oriented
%  affine isomorphisms of $\bR$, and for a family $(G_t)_{t > 0}$
%  of c.d.f.s, \ldots.
%
%
%  and any $t > 0$, we have
%  we have
%  \begin{align*}
%    A_n.F^n \tendsinlaw G_t ,
%  \end{align*}
%  where $G_t$ is a nondegenerate c.d.f.
%
%  Then, for any $t > 0$, denoting by $G^t$ the c.d.f. given by
%  $G^t(x) = \big( G(x) \big)^t$, we have
%  \begin{align*}
%    A_{n}.F^{\lfloor n t \rfloor} \tendsinlaw G^t .
%  \end{align*}
%\end{lemma}

%%\begin{lemma}[Functional equation in one parameter subgroups of affine isomorphisms]
%%  \label{lem:affine-one-parameter-subgroup}
%%  \uses{def:lem:oriented-affine-action-on-cdf}
%%  % \lean{}
%%  % \leanok
%%  Suppose that $(A_t)_{t > 0}$ is a family of affine isomorphisms of $\bR$
%%  such that
%%  \begin{align*}
%%    A_{s t} = A_s \circ A_t  \qquad \text{ for any } s, t > 0 .
%%  \end{align*}
%%  Write $A_t(x) = a_t x + b_t$, with $a_t > 0$ and $b_t \in \bR$.
%%  Then we have, for any $s, t > 0$,
%%  \begin{align*}
%%    a_{t s} = \; & a_t \, a_s
%%      \qquad \text{ and } \\
%%    b_{t s} = \; & a_t \, b_s + b_t .
%%  \end{align*}
%%  (Also by symmetry $b_{t s} = a_s \, b_t + b_s$.)
%%\end{lemma}
%%\begin{proof}
%%  % \uses{}
%%  % \leanok
%%\end{proof}
%%
%%\begin{lemma}[Functional equation scaling coefficient solution]
%%  \label{lem:solution-functional-eqn-scaling}
%%  %\uses{def:lem:oriented-affine-action-on-cdf}
%%  % \lean{}
%%  % \leanok
%%  Suppose that $a \colon (0,+\infty) \to (0,+\infty)$ is a
%%  measurable function satisfying, for any $s, t > 0$,
%%  \begin{align*}
%%    a(t s) = \; & a(t) \, a(s) .
%%  \end{align*}
%%  Then there exists a $\theta \in \bR$ such that for all $t > 0$,
%%  \begin{align*}
%%    a(t) = t^{\theta} .
%%  \end{align*}
%%\end{lemma}
%%\begin{proof}
%%  % \uses{}
%%  % \leanok
%%\end{proof}
%%
%%\begin{lemma}[Functional equation translation coefficient solution with $\theta = 0$]
%%  \label{lem:solution-functional-eqn-translation-no-scaling}
%%  %\uses{def:lem:oriented-affine-action-on-cdf}
%%  % \lean{}
%%  % \leanok
%%  Suppose that
%%  $b \colon (0,+\infty) \to \bR$ is a measurable function
%%  satisfying, for any $s, t > 0$,
%%  \begin{align*}
%%    b(t s) = \; & b(s) + b(t) .
%%  \end{align*}
%%  Then there exists a constant $c$ such that for $t \in (0,+\infty)$
%%  we have
%%  \begin{align*}
%%    b(t) = \; & -c \, \log (t) .
%%  \end{align*}
%%\end{lemma}
%%\begin{proof}
%%  % \uses{}
%%  % \leanok
%%\end{proof}
%%
%%\begin{lemma}[Functional equation translation coefficient solution with $\theta \ne 0$]
%%  \label{lem:solution-functional-eqn-translation-with-scaling}
%%  %\uses{def:lem:oriented-affine-action-on-cdf}
%%  % \lean{}
%%  % \leanok
%%  Suppose that $\theta \in \bR \setminus 0$ and
%%  $b \colon (0,+\infty) \to \bR$ is a measurable function
%%  satisfying, for any $s, t > 0$,
%%  \begin{align*}
%%    b(t s) = \; & t^{\theta} \, b(s) + b(t) .
%%  \end{align*}
%%  Then there exists a constant $c$ such that for $t \in (0,+\infty) \setminus \set{1}$
%%  we have
%%  \begin{align*}
%%    b(t) = \; & c (1 - t^{-\theta}) .
%%  \end{align*}
%%\end{lemma}
%%\begin{proof}
%%  % \uses{}
%%  % \leanok
%%\end{proof}

\begin{lemma}[Self-similar continuous c.d.f. family characterization $\gamma = 0$]
  \label{lem:characterization-self-similar-family-zero-index}
  \uses{def:lem:oriented-affine-action-on-cdf, def:translation-subgroup}
  \lean{gumbel_type_of_selfSimilar_index_zero', gumbel_type_of_selfSimilar_index_zero}
  \leanok
  Suppose that $G$ is a nondegenerate c.d.f. such that
  \begin{align*}
    G^{t} = A_t . G \qquad \text{ for any } t > 0 ,
  \end{align*}
  where
  \begin{align*}
    A_t(x) = x + \beta \, \log t
  \end{align*}
  with $\beta > 0$.

  Then with $d = \log \big(-\log G(0) \big)$,
  for all $x \in \bR$ we have
  \begin{align*}
    G(x) = \exp \Big( - \exp \big( -\beta^{-1} x + d \big) \Big) .
  \end{align*}
  (In particular, $G$ is of Gumbel type: there exists an $A \in \OriAffR$
  such that $G = A.\GumbelCDF$.)
\end{lemma}
\begin{proof}
  %\uses{lem:solution-functional-eqn-translation-no-scaling}
  \uses{lem:no-fixed-point-implies-translation}
  % \leanok

  Since $G$ is nondegenerate, there exists an $x_0 \in \bR$
  with $0 < G(x_0) < 1$. Write $q = -\log G(x_0) > 0$,
  so that $G(x_0) = e^{-q}$.

  For $t > 0$, from the equation $G^{t} = A_t . G$
  we get for any $x \in \bR$
  \begin{align*}
  G(x)^t = G \big( A_t^{-1}(x) \big) = G \big( x - \beta \, \log(t) \big) .
  \end{align*}
  In particular, with $x = x_0 + \beta \, \log(t)$, we get
  \begin{align*}
    G\big( x_0 + \beta \, \log(t) \big)^{t} = G(x_0) = e^{- q},
  \end{align*}
  from which we can solve
  \begin{align*}
    G\big( x_0 + \beta \, \log(t) \big) = e^{-q/t} .
  \end{align*}
  The above holds for any $t>0$, and any $x \in \bR$
  can be written as $x = x_0 + \beta \, \log\big( e^{(x - x_0)/\beta} \big)$.
  We therefore get, for any $x \in \bR$,
  \begin{align*}
    G(x) = \exp \Big( - q \, \exp \big( -(x-x_0)/\beta \big) \Big) .
  \end{align*}
  This is of the desired form, with $d = \frac{x_0}{\beta} + \log(q)$.
  Plugging in $x = 0$ then shows $d = \log \big(-\log G(0) \big)$.
\end{proof}

\begin{lemma}[Self-similar continuous c.d.f. family characterization $\gamma > 0$]
  \label{lem:characterization-self-similar-family-pos-index}
  \uses{def:lem:oriented-affine-action-on-cdf, def:fixing-subgroup}
  \lean{frechet_type_of_selfSimilar_index_pos', frechet_type_of_selfSimilar_index_pos}
  \leanok
  Suppose that $G$ is a nondegenerate c.d.f. such that
  \begin{align*}
    G^{t} = A_t . G \qquad \text{ for any } t > 0 ,
  \end{align*}
  where
  \begin{align*}
    A_t(x) = t^{\alpha} x + c \, (1 - t^{\alpha}) = t^{\alpha} (x-c) + c
  \end{align*}
  with $c \in \bR$ and $\alpha > 0$.

  Then with $\sigma = \big(- \log G(c+1)\big)^{\alpha}$,
  for all $x \ge c$ we have
  \begin{align*}
    G(x) = \exp \Big( - \big( \frac{x - c}{\sigma} \big)^{-1/\alpha} \big) \Big) .
  \end{align*}

  (It easily follows that $G$ is of Fr\'echet type: there exists
  an $A \in \OriAffR$ such that $G = A.\FrechetCDF{\alpha}$.)
\end{lemma}
\begin{proof}
  %\uses{lem:solution-functional-eqn-translation-with-scaling}
  \uses{lem:fixing-subgroup-characterization, lem:cdf-equal-on-dense}
  % \leanok

  Since $G$ is nondegenerate, there exists an $x_0 \in \bR$
  with $0 < G(x_0) < 1$. Write $q = -\log G(x_0) > 0$,
  so that $G(x_0) = e^{-q}$.

  %Note that
  %\begin{align*}
  %  A_t^{-1}(x) = t^{-\alpha} x + c (1 - t^{-\alpha}) = t^{-\alpha} (x-c) + c .
  %\end{align*}
  For $t > 0$, from the equation $G^{t} = A_t . G$
  we get for any $x \in \bR$
  \begin{align*}
  G(x)^t = G \big( A_t^{-1}(x) \big) . %= G \big( t^{-\alpha} (x-c) + c \big) .
  \end{align*}

  Note that for $x \le c$ and $t = 2$ we have $A_2^{-1}(x) = 2^{-\alpha}(x-c) + c \ge x$,
  so the above implies $G(x)^2 = G(A_2^{-1}(x)) \ge G(x)$.
  This not possible if $0 < G(x) < 1$, so we must in particular have $x_0 > c$.

  With $x = A_t(x_0)$ in the above equation, %= t^{\alpha} (x_0-c) + c$,
  we get
  \begin{align*}
    G\big( A_t(x_0) \big)^{t} = G(x_0) = e^{- q},
  \end{align*}
  from which we can solve
  \begin{align*}
    G(x) = G\big( A_t(x_0) \big) = e^{-q/t} .
  \end{align*}

  %The above holds for any $t>0$.
  Any $x \ge c$ can be written as $x = A_t(x_0) = t^{\alpha} (x_0-c) + c$
  with $t = \big( \frac{x - c}{x_0 - c} \big)^{1/\alpha}$
  (recall that $x_0 - c > 0$).
  We therefore get, for any $x \ge c$,
  \begin{align*}
    G(x) = \exp \Big( - q \, \big( \frac{x - c}{x_0 - c} \big)^{-1/\alpha} \big) \Big) .
  \end{align*}
  This is of the form
  \begin{align*}
    G(x) = \exp \Big( - \big( \frac{x - c}{\sigma} \big)^{-1/\alpha} \big) \Big) ,
  \end{align*}
  and plugging in $x = c + 1$ yields $\sigma = \big(- \log G(c+1)\big)^{\alpha}$.
  %This is of the desired form, with $d = \frac{x_0}{c} + \log(q)$.
  %Plugging in $x = 0$ then shows $d = \log \big(-\log G(0) \big)$.
\end{proof}

\begin{lemma}[Self-similar continuous c.d.f. family characterization $\gamma < 0$]
  \label{lem:characterization-self-similar-family-neg-index}
  \uses{def:lem:oriented-affine-action-on-cdf, def:fixing-subgroup}
  \lean{weibull_type_of_selfSimilar_index_neg', weibull_type_of_selfSimilar_index_neg}
  \leanok
  Suppose that $G$ is a nondegenerate c.d.f. such that
  \begin{align*}
    G^{t} = A_t . G \qquad \text{ for any } t > 0 ,
  \end{align*}
  where
  \begin{align*}
    A_t(x) = t^{-\alpha} x + c \, (1 - t^{-\alpha}) = t^{-\alpha} (x-c) + c
  \end{align*}
  with $c \in \bR$ and $\alpha > 0$.

  Then with $\sigma = \big(- \log G(c-1)\big)^{-\alpha}$,
  for all $x \le c$ we have
  \begin{align*}
    G(x) = \exp \Big( - \big( \frac{c - x}{\sigma} \big)^{1/\alpha} \big) \Big) .
  \end{align*}

  (It easily follows that $G$ is of Weibull type: there exists
  an $A \in \OriAffR$ such that $G = A.\WeibullCDF{\alpha}$.)
\end{lemma}
\begin{proof}
  %\uses{lem:solution-functional-eqn-scaling,
  %  lem:solution-functional-eqn-translation-exp-zero}
  \uses{lem:fixing-subgroup-characterization, lem:cdf-equal-on-dense}
  % \leanok

  \emph{(Note: The Lean formalized statement uses the opposite sign of $\alpha$:
  it is assumed that $\alpha < 0$ and $A_t(x) = t^{+\alpha} (x-c) + c$.)}

  Since $G$ is nondegenerate, there exists an $x_0 \in \bR$
  with $0 < G(x_0) < 1$. Write $q = -\log G(x_0) > 0$,
  so that $G(x_0) = e^{-q}$.

  %Note that
  %\begin{align*}
  %  A_t^{-1}(x) = t^{-\alpha} x + c (1 - t^{-\alpha}) = t^{-\alpha} (x-c) + c .
  %\end{align*}
  For $t > 0$, from the equation $G^{t} = A_t . G$
  we get for any $x \in \bR$
  \begin{align*}
  G(x)^t = G \big( A_t^{-1}(x) \big) . %= G \big( t^{-\alpha} (x-c) + c \big) .
  \end{align*}

  Note that for $x \ge c$ and $t = 2$ we have $A_2^{-1}(x) = 2^{\alpha}(x-c) + c \ge x$,
  so the above implies $G(x)^2 = G(A_2^{-1}(x)) \ge G(x)$.
  This not possible if $0 < G(x) < 1$, so we must in particular have $x_0 < c$.

  With $x = A_t(x_0)$ in the above equation, %= t^{\alpha} (x_0-c) + c$,
  we get
  \begin{align*}
    G\big( A_t(x_0) \big)^{t} = G(x_0) = e^{- q},
  \end{align*}
  from which we can solve
  \begin{align*}
    G(x) = G\big( A_t(x_0) \big) = e^{-q/t} .
  \end{align*}

  %The above holds for any $t>0$.
  Any $x \le c$ can be written as $x = A_t(x_0) = t^{-\alpha} (x_0-c) + c$
  with $t = \big( \frac{c - x}{c - x_0} \big)^{-1/\alpha}$
  (recall that $c - x_0 > 0$).
  We therefore get, for any $x \le c$,
  \begin{align*}
    G(x) = \exp \Big( - q \, \big( \frac{c - x}{c - x_0} \big)^{1/\alpha} \big) \Big) .
  \end{align*}
  This is of the form
  \begin{align*}
    G(x) = \exp \Big( - \big( \frac{c - x}{\sigma} \big)^{1/\alpha} \big) \Big) ,
  \end{align*}
  and plugging in $x = c - 1$ yields $\sigma = \big(- \log G(c-1)\big)^{-\alpha}$.
  %This is of the desired form, with $d = \frac{x_0}{c} + \log(q)$.
  %Plugging in $x = 0$ then shows $d = \log \big(-\log G(0) \big)$.
\end{proof}


\begin{theorem}[Three types of extreme value distributions [Fisher-Tippett-Gnedenko]]
  \label{thm:three-types-of-extr-val-distr}
  \uses{def:extr-val-distr,
    def:std-Gumbel-cdf, def:std-Frechet-cdf, def:std-Weibull-cdf}
  \lean{CumulativeDistributionFunction.IsExtremeValueDistr.classification}
  \leanok
  For any extreme value distribution $G$, one of the following holds:
  \begin{description}
    \item{($\GumbelCDF$)}
      $G = A . \GumbelCDF$ for some $A \in \OriAffR$;
    \item{($\FrechetCDF{{}}$)}
      $G = A . \FrechetCDF{\alpha}$ for some $A \in \OriAffR$ and $\alpha > 0$;
    \item{($\WeibullCDF{{}}$)}
      $G = A . \WeibullCDF{\alpha}$ for some $A \in \OriAffR$ and $\alpha > 0$.
  \end{description}
  In particular, the only three possible types of extreme value distributions
  are the type of the Gumbel c.d.f., the type of
  the Fr\'echet c.d.f. $\FrechetCDF{\alpha}$ for $\alpha > 0$, and
  the type of the Weibull c.d.f. $\WeibullCDF{\alpha}$ for $\alpha > 0$.
\end{theorem}
\begin{proof}
  \uses{thm:one-parameter-subgroups-of-affine-isomorphisms,
    lem:self-similarity-of-extreme-value-distributions,
    lem:characterization-self-similar-family-zero-index,
    lem:characterization-self-similar-family-pos-index,
    lem:characterization-self-similar-family-neg-index}
  \leanok
  \ldots
\end{proof}
